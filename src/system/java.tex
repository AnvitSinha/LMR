\paragraph{Import}
Libraries that are commonly used.
\begin{lstlisting}[language=java]
import java.io.*;
import java.lang.*;
import java.math.*;
import java.util.*;
\end{lstlisting}

\paragraph{Input}
\texttt{Scanner} is generally used to handle input.
\begin{lstlisting}[language=java]
Scanner in = new Scanner (System.in);
\end{lstlisting}
Or:
\begin{lstlisting}[language=java]
Scanner in = new Scanner (new BufferedInputStream (System.in));
\end{lstlisting}
Usage: \texttt{next + <typename> ()}, \texttt{hasNext + <typename> ()}.

e.g. \texttt{in.nextInt ()}, \texttt{in.nextBigInteger ()}, \texttt{in.nextLine ()}, \texttt{in.hasNextInt ()}, etc.

\paragraph{Output}
Use \texttt{System.out} for output.
\begin{lstlisting}[language=java]
System.out.print (/*...*/);
System.out.println (/*...*/);
System.out.printf (/*...*/);
\end{lstlisting}

\paragraph{BigInteger}
To convert to a \texttt{BigInteger}, use \texttt{BigInteger.valueOf (int)} or \texttt{BigInteger (String, radix)}.

To convert from a \texttt{BigInteger}, use \texttt{.intValue ()}, \texttt{.longValue ()}, \texttt{.toString (radix)}.

Common unary operations include \texttt{.abs ()}, \texttt{.negate ()}, \texttt{.not ()}.

Common binary operations include \texttt{.max}, \texttt{.min}, \texttt{.add}, \texttt{.subtract}, \texttt{.multiply}, \texttt{.divide}, \texttt{.remainder}, \texttt{.gcd}, \texttt{.modInverse}, \texttt{.and}, \texttt{.or}, \texttt{.xor}, \texttt{.shiftLeft (int)}, \texttt{.shiftRight (int)}, \texttt{.pow (int)}, \texttt{.compareTo}.

Divide and remainder: \texttt{Biginteger[] .divideAndRemainder (Biginteger val)}.

Power module: \texttt{.modPow (BigInteger exponent, module)}.

Primality check: \texttt{.isProbablePrime (int certainty)}.

Square root:
\begin{lstlisting}[language=java]
public static BigInteger sqrt (BigInteger x) {
	if (x.equals (BigInteger.ZERO) || x.equals (BigInteger.ONE)) return x;
	BigInteger d = BigInteger.ZERO.setBit (x.bitLength () / 2);
	BigInteger d2 = d;
	for (; ; ) {
		BigInteger y = d.add (x.divide (d)).shiftRight (1);
		if (y.equals (d) || y.equals (d2)) return d.min (d2);
		d2 = d; d = y; } }
\end{lstlisting}

\paragraph{BigDecimal}
Literally a \texttt{BigInteger} and a scale.

When rounding, it is necessary to specify a \texttt{RoundingMode}, namely \texttt{RoundingMode.<mode>}, which includes: 

\texttt{CEILING}, \texttt{DOWN}, \texttt{FLOOR}, \texttt{HALF\_DOWN}, \texttt{HALF\_EVEN}, \texttt{HALF\_UP}, \texttt{UNNECESSARY}, \texttt{UP}.

To convert to a \texttt{BigDecimal}, use \texttt{BigDecimal.valueOf (...)}, \texttt{BigDecimal (BigInteger, scale)} or \texttt{BigDecimal (String)}.

To divide: \texttt{.divide (BigDecimal, scale, roundingmode)}.

To set the scale: \texttt{.setScale (scale, roundingmode)}.

To remove trailing zeroes: \texttt{.stripTrailingZeros ()}.

\paragraph{Array}
Sort: \texttt{Arrays.sort (T[] a);}

\texttt{Arrays.sort (T[] a, int fromIndex, int toIndex);}

\texttt{Arrays.sort (T[] a, int fromIndex, int toIndex, Comperator <? super T> comperator);}

\paragraph{PriorityQueue}
An implementation of a min-heap.

Add element: \texttt{add (E)}.

Retrieve and pop element: \texttt{poll ()}.

Retrieve element: \texttt{peek ()}.

Size: \texttt{size ()}.

Clear: \texttt{clear ()}.

Comparator: \texttt{PriorityQueue <E> (int initcap, Comparator <? super E> comparator)}

\paragraph{TreeMap}
An implementation of a map. The entry is named \texttt{Map.Entry <K, V>}.

Retrieve key and value from an entry: \texttt{getKey}, \texttt{getValue ()}, \texttt{setValue (V)}.

Retrieve entry: \texttt{ceilingEntry}, \texttt{floorEntry}, \texttt{higherEntry}, \texttt{lowerEntry}.

Simplified operations: \texttt{clear ()}, \texttt{put (K, V)}, \texttt{get (K)}, \texttt{remove (K)}, \texttt{size ()}.

Comparator: \texttt{TreeMap <K, V> (Comparator <? super K> comparator)}.

\paragraph{StringBuilder}

Construction: \texttt{StringBuilder (String)}.

Insertion: \texttt{append (...)}, \texttt{insert (offset, ...)}. \texttt{...} can be almost every type!

Fetch: \texttt{charAt (int)}.

Modification: \texttt{setCharAt (int, char)}, \texttt{delete (int, int)}, \texttt{reverse ()}.

Output: \texttt{length ()}, \texttt{toString ()}.

\paragraph{String}
Formatting: \texttt{String.format (String, ...)}.

Case transform: \texttt{toLowerCase}, \texttt{toUpperCase}.

\paragraph{Comparator}
An example on a comparator.
\begin{lstlisting}[language=java]
public class Main {
	public class Point {
		public int x; public int y;
		public Point () {
			x = 0;
			y = 0; }
		public Point (int xx, int yy) {
			x = xx;
			y = yy; } }
	public class Cmp implements Comparator <Point> {
		public int compare (Point a, Point b) {
			if (a.x < b.x) return -1;
			if (a.x == b.x) {
				if (a.y < b.y) return -1;
				if (a.y == b.y) return 0; }
			return 1; } }
	public static void main (String [] args) {
		Cmp c = new Cmp ();
		TreeMap <Point, Point> t = new TreeMap <Point, Point> (c);
		return; } }
\end{lstlisting}

\paragraph{Comparable}
An example to implement \texttt{Comparable}.
\begin{lstlisting}[language=java]
public class Point implements Comparable <Point> {
	public int x; public int y;
	public Point () {
		x = 0;
		y = 0; }
	public Point (int xx, int yy) {
		x = xx;
		y = yy; }
	public int compareTo (Point p) {
		if (x < p.x) return -1;
		if (x == p.x) {
			if (y < p.y) return -1;
			if (y == p.y) return 0; }
		return 1; }
	public boolean equalTo (Point p) {
		return (x == p.x && y == p.y); }
	public int hashCode () {
		return x + y; } }
\end{lstlisting}

\paragraph{Fast IO}
A class for faster IO.
\begin{lstlisting}[language=java]
public class Main {
	static class InputReader {
		public BufferedReader reader;
		public StringTokenizer tokenizer;
		public InputReader (InputStream stream) {
			reader = new BufferedReader (new InputStreamReader (stream), 32768);
			tokenizer = null; }
		public String next() {
			while (tokenizer == null || !tokenizer.hasMoreTokens()) {
				try {
					String line = reader.readLine();
					tokenizer = new StringTokenizer (line);
				} catch (IOException e) {
					throw new RuntimeException (e); } }
			return tokenizer.nextToken(); }
		public BigInteger nextBigInteger() {
			return new BigInteger (next (), 10); /* radix */ }
		public int nextInt() {
			return Integer.parseInt (next()); }
		public double nextDouble() {
			return Double.parseDouble (next()); } }
	public static void main (String[] args) {
		InputReader in = new InputReader (System.in);
	} }
\end{lstlisting}
