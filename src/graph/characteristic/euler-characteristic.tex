The Euler characteristic $\chi$ was classically defined for the surfaces of polyhedra, according to the formula
$$\chi =V-E+F$$
where $V$, $E$, and $F$ are respectively the numbers of vertices (corners), edges and faces in the given polyhedron. Any convex polyhedron's surface has Euler characteristic
$$V-E+F=2\,.$$

This equation is known as Euler's polyhedron formula. It corresponds to the Euler characteristic of the sphere (i.e. $\chi = 2$), and applies identically to spherical polyhedra.

The Euler characteristic of a closed orientable surface can be calculated from its genus $g$ (the number of tori in a connected sum decomposition of the surface; intuitively, the number of "handles") as
$$\chi=2-2g\,.$$
The Euler characteristic of a closed non-orientable surface can be calculated from its non-orientable genus $k$ (the number of real projective planes in a connected sum decomposition of the surface) as
$$\chi=2-k\,.$$

Euler's formula also states that if a finite, connected, planar graph is drawn in the plane without any edge intersections, and $v$ is the number of vertices, $e$ is the number of edges and $f$ is the number of faces (regions bounded by edges, including the outer, infinitely large region), then
$$v-e+f=2\,.$$

In a finite, connected, simple, planar graph, any face (except possibly the outer one) is bounded by at least three edges and every edge touches at most two faces; using Euler's formula, one can then show that these graphs are sparse in the sense that if $v \ge 3$:
$$e\leq 3v-6\,.$$ 

